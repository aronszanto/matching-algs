\documentclass[12pt]{article}
\usepackage{geometry}
\geometry{letterpaper}
\usepackage{graphicx}
\usepackage{amsmath, amssymb, amsthm}
\usepackage{relsize}
\usepackage{fancyhdr}
\usepackage{listings}
\usepackage{color}

\lstset{
  language=Python,
  aboveskip=3mm,
  belowskip=3mm,
  showstringspaces=false,
  columns=flexible,
  basicstyle={\small\ttfamily},
  numbers=left,
  breaklines=true,
  breakatwhitespace=true,
  tabsize=2
}
\providecommand{\ceil}[1]{\left \lceil #1 \right \rceil }
\providecommand{\floor}[1]{\left \lfloor #1 \right \rfloor }

\newcommand{\noin}{\noindent}

\newcommand{\studentname}{Aron Szanto & Jerry Anunrojwong}
\newcommand{\exerciseset}{Final Project}

\fancypagestyle{plain}{}
\pagestyle{fancy}
\fancyhf{}
\fancyfoot[RO,LE]{\sffamily\bfseries\thepage}
\renewcommand{\headrulewidth}{1pt}
\renewcommand{\footrulewidth}{1pt}

\graphicspath{{figures/}}

\title{CS 136 \exerciseset}
\author{\studentname}

\begin{document}
\maketitle

%%%%%%%%%%%%%%%%%%%%%
\section*{Introduction}
In this project, we consider the housing market problem from a variety of perspectives, both theoretical and practical. In the simplest formulation of this classic problem, a set $N$ of agents owns a set $H$ of houses, and the goal is to find a matching that is Pareto-optimal. Roughly speaking, this implies that under the matching scheme, there is no deviation benefiting an agent $i$ that would not decrease the utility of an agent $j \neq i$, for all $i$ and $j$.\\\\
The problem becomes more interesting when a certain amount of leeway is granted with respect to the parameters of possible solutions. By opening the door to probabilistic algorithms, tweaks to the initial ownership schema, and permutations of preference generation, some very interesting results can be derived, especially when the metrics used to discuss solutions are broadened beyond simple Pareto-optimality.\\\\For our project, we were interested in programming three of the most important algorithms associated with the house allocation problem, in addition to developing software to analyze the efficiency and quality of the solutions output. In particular, we created programs in the python language to model the Top Trading Cycles, You Request My House I Get Your Turn, and Probabilistic Serial algorithms, hereafter referred to simply by their (perhaps truncated) initials (TTC, YRM, PS, respectively). In addition, we analyzed the solutions returned by these procedures via metrics including ex-post Pareto-optimality, time complexity (performance speed), and envy-freeness. Last, we played with preference generation, considering the Mallow model but settling on a scheme relying upon cardinal utilities modeled from a hierarchical normal distribution, guaranteeing correlated preferences. While our writeup will highlight our findings and showcase the most important pieces of our project, we stress that the brunt of the effort expended on this collaboration went into an extensive implementation of the several algorithms and solution metrics, the entirety of which is included with this paper. At certain points in the paper, we will direct the reader to run the scripts associated with certain files if so desired, to see results firsthand. In addition, we will often provide sample output within the text so as to avoid extensive context-switching.\\\\In the first section, we will discuss the preference generation techniques utilized to parametrize our agents' utilities. Next, we will give an overview of the algorithms that we coded. Following that, we detail the metrics used to analyze the quality of the solutions we generate. We then move to our experimental results, showing sample inputs and outputs to our algorithms. Ultimately, we discuss the challenges we faced and address some of the reasoning behind our design decisions in our conclusion.
%%%%%%%%%%%%%%%%%%%%%
\section*{Preferences}
At the heart of all matching problems is a preference scheme that dictates each agent's preferences. In particular, a preference system must be able to determine, given two alternatives, which, if either, is preferred. In this instance, we model pairwise preference as a cardinal comparison between the utilities afforded to an agent by the two alternatives proposed. More formally, an agent $i$'s utility over alternative $a$ in set $A$ is given by a function $u_i: A \rightarrow {\Bbb R}$, and an operator such as $\succ$ is defined (loosely) by the equivalence $u_i(a_1) > u_i(a_2) \equiv a_1 \succ_i a_2$. In relation to our problem, the preference operator denotes that an agent prefers the assignment of a particular house (or other item, though housing is the canonical example in the literature) to that of another.\\\\
Thus, the way in which we model preferences inherently dictates the scope, direction, and results of our simulation. One option we explored early on (at the suggestion of Debmalya Mandal) was the Mallow model, a way of generating ordinal (and thus pairwise) preferences across a set of alternatives, clasically parametrized by a dispersion factor $\phi$ that relates the "closeness" of agents' preferences. In other words, dispersion is the degree to which there is an underlying "true" quality assessment that each agent tends to agree with. However, we decided against this model because preferences in the real world are poorly modeled by way of randomly selecting alternatives to be better than others, even if there is a system to the randomness. Under the Mallow model, there is no way to control the degree to which a house is more inherently desireable than another. However, a system of two-tiered normally distributed preferences allows a great deal of control over both the underlying quality distribution of houses as well as the degree to which agents' preferences take into account this distribution.\\\\We model preferences as follows. Build a vector $V \in {\Bbb R^{|N|}}$ by sampling from a normal distribution with mean $\mu$ and variance $\sigma^2$. For each $v_i$ in $V$, create a normal distribution with mean $v_i$ and variance $\eta^2$. For each agent, sample their utility for assignment of house $i$ from the secondary distribution. In this way, one can fine-tune the way preferences correlate between agents. On one side, increasing $\sigma$ spreads out the underlying qualities of the houses, making preferences more distinct, and thus less independent. On the other, increasing $\eta$ makes differences in underlying quality less significant in determining pairwise preferences. Providing a snippet:
\begin{verbatim}
def assign_preferences(n=config.NUM_AGENTS):

    # create means
    item_means = [Distribution(config.ITEM_MEAN, config.ITEM_VAR).sample()
                  for _ in xrange(n)]

    # create agents and shuffle their order
    agents = [Agent(i) for i in xrange(n)]
    random.shuffle(agents)

    # assign agents preferences

    for agent in agents:
        agent.cardinal_prefs = [Distribution(
            item_mean, config.PREFERENCE_VAR).sample() for item_mean in item_means]

    # sort agents' preferences
    for agent in agents:
        agent.ordinal_prefs = [sorted(agent.cardinal_prefs).index(x)
                               for x in agent.cardinal_prefs]
\end{verbatim}

\noindent When run with standard configuration 
\begin{verbatim}
ITEM_VAR = 63 # sigma^2
ITEM_MEAN = 50 # v_i
PREFERENCE_VAR = 44.7 # eta
NUM_AGENTS = 100
\end{verbatim}
\noindent the correlation between two agents' preferences is around 0.7, which we determined to be a sound basic representation of truth (see line 30 of sim.py)
%%%%%%%%%%%%%%%%%%%%%
\section*{Algorithms}

%%%%%%%%%%%%%%%%%%%%%
\section*{Metrics}
%%%%%%%%%%%%%%%%%%%%%
\section*{Experimental Results}

%%%%%%%%%%%%%%%%%%%%%
\section*{Conclusion}
%%%%%%%%%%%%%%%%%%%%%

\end{document}